% ---
% EXEMPLO PARA CURSO DE POS GRADUAÇÃO
% ---

% \instituicao{Instituto Militar de Engenharia}
% \programapgdepartamento{Engenharia de Transportes}
% \nivelestudo{Doutorado} % Graduação, Mestrado ou Doutorado
% % ---
% \titulo{Modelo Canônico de Trabalho Acadêmico com \abnTeX\space \versaoDocumento}
% \palavraschave{arp,sarp,iot,vant,tarefas cooperativas,agentes inteligentes}
% \keywords{unmanned systems,unmanned vehicles,uav,uas,cooperative tasks,intelligent agents}
% % ---
% \autores{Fulano de}{Silva}% 1+ autores
% % \autor{Fulano de}{Tal} %{nome}{sobrenome}
% \orientadores{Sicrano,Beltrano}{Santos,Oliveira}{Ph.D.,D.Sc.}%{nomes}{sobrenomes}{títulos}
% % ---
% \local{Rio de Janeiro}
% \data{2020}
% \datadefesa{30 de fevereiro de 2020}
% \bancadeexaminadores{
%     Prof. \textbf{Orientador 1} - D.Sc. do IME - Presidente,
%     Prof. \textbf{Orientador 2} - D.Sc. do LNCC,
%     Prof. \textbf{Professor 1} - Ph.D. do IMPA,
%     Prof. \textbf{Professor 2} - D.Sc. do LNCC,
%     Prof. \textbf{Professor 3} - D.Sc. do IME,
%     Prof. \textbf{Professor 4} - D.Sc. da PUC
% }

% ---



% ---
% EXEMPLO PARA PROJETO DE FIM DE CURSO
% ---

\instituicao{Instituto Militar de Engenharia}
\programapgdepartamento{Engenharia Mecânica}
\nivelestudo{Mestrado} % Graduação, Mestrado ou Doutorado
% ---
\titulo{Análise aerodinâmica e estudo do alcance de um projétil de artilharia}
\palavraschave{aerodinâmica,força de arrasto,extensão de alcance,trajetória balística,fluidodinâmica computacional}
\keywords{aerodynamics,drag force,range extension,ballistic trajectory,computational fluid dynamics}
% ---
\autores{Wallace}{Ramos Rosendo da Silva}% {nome}{sobrenome} 1+
\orientadores{André}{Luiz Tenório Rezende}{D.Sc.}%{nomes}{sobrenomes}{títulos} 1+
% ---
\local{Rio de Janeiro}
\data{2023}
\datadefesa{\today}
\bancadeexaminadores{
    Prof. \textbf{André Luiz Tenório Rezende} - D.Sc. do IME - Presidente,
    Prof. \textbf{Victor Santoro Santiago} - D.Sc. do IME,
    Prof. \textbf{Bruna Rafaella Loiola} - D.Sc. do IME
}

% ---