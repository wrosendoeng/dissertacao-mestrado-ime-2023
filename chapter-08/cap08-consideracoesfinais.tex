\chapter{Comentários Finais}\label{cap:conclusoes}

O trabalho iniciou a partir da pesquisa bibliográfica sobre aerodinâmica de um projétil de artilharia. A partir deste ponto, duas regiões de domínio foram modeladas sob as mesmas condições de contorno. Definindo-se estas condições e modificando o regime de velocidade do escoamento, foram realizados testes com CFD a partir dos modelos RANS de turbulência Spalart-Allmaras \cite{Spalart1992} e SST $\kappa$-$\omega$ \cite{Menter1994TwoequationET,Menter2003,Menter2009}, considerando ou não o efeito \textit{Base Bleed}.

Como se trabalha com fenômenos de turbulência aplicados em escoamentos compressíveis, algumas simplificações foram implementadas nas simulações CFD: domínio bidimensional e axissimétrico, conforme as principais referências \cite{Mahmoud2009,Lucena2020}; regime permanente, logo não se analisou a influência do tempo no escoamento; fluido como gás ideal para fechar o algoritmo \textit{density-based} com as equações de gases perfeitos para a pressão em função da densidade e temperatura e a viscosidade pela lei de Sutherland.

As primeiras soluções para os modelos SST $\kappa-\omega$ foram baseadas no projetil inerte (sem \textit{Base Bleed}) para análise de convergência, pautados em valores de referência \cite{Mahmoud2009} do coeficiente de arrasto aerodinâmico para um projetil de calibre 155mm. Duas malhas foram testadas sob o mesmo domínio computacional, a fim de analisar qualitativamente e quantitavimente os resultados de acordo com o número de elementos. Só então que foram verificadas as influências do diâmetro de injeção, \(\phi_{BB}\); da temperatura dos gases na saída do bocal, \(T_{BB}\), e da vazão do sistema \textit{Base Bleed}, \(\Dot{m}_{BB}\). Para o modelo Spalart-Allmaras (S-A), a verificação foi feita apenas com projetil ativo (com \textit{Base Bleed}) com diâmetro e vazão do sistema \textit{Base Bleed} fixados para verificar a diferença dos resultados entre as metodologias S-A e SST $\kappa-\omega$.

Utilizou-se o programa Fluent \cite{fluent2021ansys} para aplicação dos modelos de turbulência e respectivas soluções. A abordagem de discretização do código é baseada no Método de Volumes Finitos \cite{McDonald1971,MacComarck&Paulay1972}. Todos os casos supracitados fizeram uso do esquema \textit{upwind} de segunda ordem para a discretização dos termos advectivos e difusivos. Para os gradientes das propriedades foi escolhido o método dos Mínimos Quadrados por ser um método mais barato computacionalmente \cite{fluent2021ansys}. Em razão do regime de voo do projetil, o algoritmo \textit{density-based} \cite{Weiss1995PreconditioningAT,Weiss1997IMPLICITSO,Weiss1999ImplicitSO} foi aplicado em razão da compressibilidade. Para resolução do sistemas algébricos gerados em cada simulação CFD o método \textit{multigrid} \cite{Hutchinson1986} foi selecionado. A conjunção dessas técnicas garantiu os resultados nos dois modelos RANS selecionados.

Os valores encontrados pelas simulações CFD possibilitaram a aplicação do modelo de trajetória ponto-massa modificado (MPMTM), padronização organizada pela OTAN \cite{stanag4355} e aplicada para projetis axissimétrica estabilizados pela rotação. Esse modelo tem o foco em reduzir custos computacionais quando comparado ao modelo de dinâmica de corpo rígido, inclusive para operações de combate, fazendo dele o mais indicado para elaboração de tabela de tiros de munições. O código desenvolvido em MATLAB\textregistered{} \cite{ThallyoENCIT2022,Thallyo2022} calculou as condições da munição 155mm disparadas em 2 ângulos de elevação diferentes; com e sem \textit{Base Bleed}; assim como a variação de parâmetros nos casos em que disparou o projetil ativo: o diâmetro de saída do bocal de injeção e a vazão mássica do propelente. 

\section{Conclusão}

Os resultados para as simulações de fluidodinâmica computacional mostraram uma tendência positiva no que se refere aos valores do coeficiente de arrasto, tendo em vista que a implementação da tecnologia \textit{Base Bleed} implica em mexer em parâmetros sensíveis, inclusive de design da munição. A abordagem deste estudo foi desenvolver estimativas iniciais, mesmo sem formulações químicas realísticas para simular a queima do propelente atuante na região a jusante do projetil. Contudo, os parâmetros avaliados como o diâmetro de saída do bocal de injeção, a temperatura e a vazão mostraram influência significativa nos resultados finais.

O modelo SST $\kappa-\omega$ conseguiu predizer os valores do $C_D$ para o projetil inerte com certa razoabilidade, assim como descrito nas referências \cite{Mahmoud2009,nicolas-perez_accuracy_2017}. Contudo, a interação entre a frente de chama do \textit{Base Bleed} e a esteira turbulenta presente na base do projetil dificulta as estimativas com maior acurácia usando modelos RANS \cite{nicolas-perez_accuracy_2017}. Em seu estudo, \citeauthor{Spalart1992} afirma que é difícil modelar um choque quando há um alto gradiente adverso de pressão, além de outros colaboradores da pesquisa deste artigo mencionarem que são produzidos resultados de baixa qualidade em escoamentos que há um degrau descendente (\textit{backward-facing step}) por causa das tensões produzidas. Este argumento também pode explicar a falta de precisão dos resultados na região após o \textit{boat-tail}.

Conforme visto nas referências, o regime transônico apresentou os valores mais elevados para o coeficiente de arrasto, enquanto o regime supersônico relatou uma queda mais gradual neste coeficiente com o aumento do número de Mach, com algumas singularidades ao mudar certas variáveis. O parâmetro mais relevante para mudar o arrasto de base foi o diâmetro de saída do bocal, apesar de tanto a vazão mássica e a temperatura mostrarem uma redução.

A implementação do programa para o cálculo da trajetória com 4 graus de liberdade foi feita sem considerar os efeitos de ignição e com vazão constante do \textit{Base Bleed} durante toda a queima, o que não é um modelo realístico, tendo em vista que a taxa de queima de propelente tem grande influência da rotação e da pressão atmosférica. Como a geometria do projetil foi simplificada para se ter estimativas do $C_D$, a massa da câmara foi considerada apenas o conteúdo do propelente. Sabe-se que em projetis de artilharia a base contém cavidade, fora os componentes adicionados para fazer a propulsão sólida que adicionam massa e mudam a estrutura.

\section{Recomendações}

Espera-se em estudos futuros que novos fatores geométricos sejam considerados em projeto, como o design da câmara geradora de gás acoplada à munição; os efeitos de rotação do projetil; a contribuição do sistema de ignição para a tecnologia \textit{Base Bleed} e implementação de possíveis formulações para o propelente a ser inserido na câmara geradora de gás, já que assumir os gases em combustão como perfeitos é uma simplificação para garantir um ponto de partida do projeto. 

Em se tratando de simulações numéricas, recomenda-se implementar modelos de turbulência LES ou DNS para captar a mistura do jato do \textit{Base Bleed} com a esteira produzida no escoamento na base da munição, seguindo a linha das principais referências bibliográficas \cite{nicolas-perez_accuracy_2017,Lucena2020}. A simplificação da malha bidimensional e axissimétrica só foi possível pois não se estudou a influência do ângulo de ataque nos coeficientes aerodinâmicos. Portanto, um caminho natural para esta análise seria desenvolver uma malha tridimensional.

Com os avanços tecnológicos, espera-se que simulações com processamento em placa gráfica (GPU) possam ser utilizados para otimização dos resultados e permitindo análises mais refinadas, inclusive com malhas com um número maior de elementos e modelos de turbulência mais complexos do que os baseados em RANS.

Sobre o modelo de trajetória, um estudo de eficiência computacional para limpeza do código seria um passo concreto para otimização dos processos de predição de voo. O uso de uma linguagem de programação compilada além do MATLAB\textregistered{} também seria interessante para aumentar a velocidade de processamento, já que se trata de uma linguagem interpretada e por ser um produto comercial, tem restrições de uso para desenvolver um programa que contemple os interesses de pesquisa e de transferência tecnológica para o Exército Brasileiro.

Finalmente, os estudos desenvolvidos neste projeto produziram conteúdos para publicação em congressos científicos, dentre os quais 1 publicação para a Revista Militar de Ciência e Tecnologia (RMCT) e 4 artigos para congressos de Engenharia realizados em 2022 \cite{RosendoCBCFD2022,ThallyoENCIT2022,RosendoCONEM2022,BernardoCONEM2022,RosendoCILAMCE2022}. Acredita-se que esforços futuros em novas simulações numéricas permitam a esta linha de pesquisa um potencial de desenvolvimento tecnológico de interesse para o Exército Brasileiro. 